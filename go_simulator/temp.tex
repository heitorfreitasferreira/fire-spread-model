\documentclass{article}
\usepackage{amsmath}
\usepackage{amsfonts}

\begin{document}

\section*{Matriz Original e Matriz Expandida}

A matriz original \( M_{\text{original}} \) é dada por:
\[
\begin{bmatrix} 
0.07 & 0.1 & 0.13 \\
0.4 & 0 &  0.16\\
0.07 & 0.1 & 0.13
\end{bmatrix}
\quad
\begin{bmatrix}
0.025 & 0.035 & 0.05 & 0.065 & 0.046 \\
0.35 & 0.07 & 0.1 & 0.13 & 0.065 \\
0.02 & 0.04 & 0 & 0.16 & 0.08 \\
0.035 & 0.07 & 0.1 & 0.13 & 0.065 \\
0.025 & 0.035 & 0.05 & 0.065 & 0.046
\end{bmatrix}
\]

\section*{Função de Expansão}

O processo de expansão da matriz segue estas etapas:

1. A matriz \( M_{\text{original}} \) é centralizada na nova matriz \( M_{\text{expandida}} \), de dimensão \( (2 \times \text{novoRaio} + 1) \times (2 \times \text{novoRaio} + 1) \), onde \( \text{novoRaio} = 2 \). A nova matriz tem tamanho \( 5 \times 5 \).

2. Preenchimento das áreas expandidas:
   - Para células na direção vertical e horizontal:
   \[
   M_{\text{expandida}}[i, j] = M_{\text{expandida}}[i', j'] \times 0.5
   \]
   onde \( (i', j') \) são as coordenadas da célula adjacente mais próxima da matriz original.

   - Para células nas diagonais:
   \[
   M_{\text{expandida}}[i, j] = M_{\text{expandida}}[i', j'] \times 0.3535
   \]
   onde \( (i', j') \) são as coordenadas da célula diagonal adjacente mais próxima da matriz original.

\section*{Decaimento}

Os fatores de decaimento aplicados são:

- **Decaimento vertical e horizontal**:
  \[
  \text{verticalHorizontalDecai} = 0.5
  \]

- **Decaimento diagonal**:
  \[
  \text{diagonalDecai} = 0.5 \times \frac{\sqrt{2}}{2} = 0.3535
  \]

\end{document}
